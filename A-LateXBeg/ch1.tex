
{   \itshape \large 
    \raggedleft
    ``Live life to the fullest, and focus on the positive.'' \\
    --- Matt Cameron \\
}
\section{Basic Introduction of \TeX{} and \LaTeX}

\TeX{} is a typesetting software program which was created by \index{Donald E.Knuth.} 
It is able to generate high-quality printable documents. The core philosophy of 
\TeX{} is ``What You See Is What You Mean (WYSIWYM)'', which makes writers more 
focus on contents. The way of using \TeX{} to make a document is somewhat similar
to programming, writing files which consist of commands and texts under 
specific grammar rules and then compiling these files to a printable document.

\begin{figure}[!htbp]
    \centering
    \begin{tikzpicture}
        \begin{loglogaxis}[
        title=Convergence Plot,
        xlabel={Degrees of freedom},
        ylabel={$L_2$ Error},
        ]
        \addplot table {data_d2.dat};
        \end{loglogaxis}
    \end{tikzpicture}
    \caption{An example picture produced by pgfplot}
    \label{fig:pgfplot}  
\end{figure}

%TODO: unset the indent
In order to make \TeX{} eaisier to use, \index{Dr.~Lamport} developed an extendable macro 
library based on the original \TeX{}, which is the \LaTeX.

%Typical work-flow of using \LaTeX


\section{Grammar Structure of \LaTeX}

%command

%text

\section{Files Organization of \LaTeX}

%main file

%input and include

\section{Encoding}
\section{Chinese Support}
\section{Brief Introduction of Symbols}
\section{Miscs}
\subsection{Text Emphasis}
\underline{This is an example of \textbackslash underline.} \\
\emph{This is an example of \textbackslash emph}
\subsection{Space, Line, Paragraph and Page Control}
\subsubsection{Spaces}
\~{}: \LaTeX{} does not break a line from the position of \~{}. 
See the difference between Prof. X and Prof.~X.

\subsubsection{Lines}
\textbf{\textbackslash newline} and \textbf{\textbackslash \textbackslash} are used
for breaking \newline lines.

\subsubsection{Paragraphs}
\textbf{\textbackslash par} and a empty line are used for starting a new 
\par 
paragraph. 


BTW, multiple successive empty lines are considered as one.

\subsubsection{Pages}
Now, use \textbf{\textbackslash newpage} or \textbf{\textbackslash clearpage} to 
open a new page.

\section{XXX}
\lipsum

Test index.
\index{Test@\textsf{""Test}|(textbf}
\index{Test@\textsf{""Test}!sub@"|sub"||see{Test}}
\newpage
Test index.
\newpage
Test index.
\index{Test@\textsf{""Test}|)textbf}
\newpage
Test index.


\newpage
