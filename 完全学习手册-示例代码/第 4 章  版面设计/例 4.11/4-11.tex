\documentclass[twocolumn]{article}
\usepackage[paperwidth=145mm,paperheight=99.5mm,text={134mm,120mm},left=6mm,top=-7mm]{geometry} % ҳ������
\usepackage{amsmath}
\renewcommand{\rmdefault}{ptm}
\begin{document}
\title{Physical Model Order Reduction}
\author{Qiang Wang and Guo-Hua Li\\[4pt]
Department of Electronic Engineering,
the East University of China}
\date{December 17, 2009}
\maketitle
\begin{abstract}
This paper presents a novel
approach for model order reduction for multilayer lossy RF
embedded passives.
\end{abstract}

\section{Introduction}
As the Radiofrequency modules having been designed more compact
than ever before, the parasitic effects due to the tightly coupled
inter-connections on the circuit layout are inevitable. Therefore,
an efficient method that can derive a circuit model of such circuit
layout is highly desirable. A few techniques such as PEEC were developed to
extract equivalent circuits from an electromagnetic model.
Classic PEEC solver converts the layout into lumped RLC
interconnection networks, including mutual couplings. Once a circuit
model is generated, any circuit solver, such as SPICE, can manage
the rest of the job. Unfortunately, the numbers of nodes and
elements in the circuits are excessive. Therefore, researchers have
been searching for an effective measure that can reduce the model
order for accelerating the analysis of the circuit model.

Although exploited Krylov
subspace methods and provided ways to speed up the simulation; they
all lack the physical insight. In fact, they are mathematics-based
MOR. Several realizable model order reduction approaches are also
proposed. However, can only handle RC networks. Besides, they both
concern with matching the first two or three moments of the system
via Taylor's expansion. Thus, these methods can not provide a clear
physical explanation to the reduced circuit either. It is worth
mentioning that has showed some insight for
dealing with coupled inductances, even though the complexity of the
scheme itself might have already limited its practical use.

The work presented in this paper is an extension to,
in which a derived physically realizable lossless expressive circuit
model reduction method is introduced. In this paper, a lossy model
is the major concern. The passivity of the resultant circuit model
by the new reduction scheme is guaranteed.

\section{Theory}
The circuit model generated by traditional quasi-static PEEC model
for a multi-layer circuit layout with very thin conducting strips
can easily incorporate the conducting loss, which is a major origin
of the circuit loss.

Since the meshes used in solving MPIE (Mixed Potential Integrated
Equations) of the PEEC algorithm are all in regular shapes, thus we
could first evaluate their losses piecewisely and then superimpose
this pre-calculated loss model to the generated circuit model to
represent the conductor loss of the circuit. Therefore reasonable
and time-saving approaches to calculate the loss for different
meshing geometries are investigated.

Since the conductor loss is generally determined by the skin depth
effect at RF frequency, a coarse but rapid approximation to this
type of loss is to find out the skin depth and other shape factors
of the mesh. Then the equivalent surface impedance can be easily
calculated by $R_{L} = l / 2\delta S$, where $l$ is the mesh length,
along which the current flows, $S$ is the area of the equivalent
crossing section where the current goes through, $\delta$ is the
skin depth:
\begin{align}
 \delta=\frac{1}{\sqrt{\smash[b]{\pi f \mu \sigma}}},
 \label{Equ1}
 \end{align}
in which $f$ is the frequency, $\mu$ is the magnetic permeability,
and $\sigma$ is the conductivity of the metal. In addition to this,
various other empirical formulas, such as those in
could also be used to
determine the conducting loss.

Capacitor's loss is mainly due to bypassing leak-age current.
Considering the basic relationship be-tween the current and the
charge stored in the capacitor, its loss
contribution could be computed by $G_{C} = \sigma C / \varepsilon$,
where $G_{C}$ is the bypassing conductance, $C$ is the capacitance,
$\varepsilon$ and $\sigma$ are the dielectric constant and the
conductivity of the substrate, respectively.

...
\end{document}
